\subsection*{Priority inversion}
Priority inversion occurs when a higher-priority task is waiting for a resource held by a
lower-priority task. The main sources of priority inversion are non preemptable sections,
sharing resources. synchronization and mutual exclusion. The solution is to use Resource
Access Protocols\\
\underline{Non-Preemptive Protocol (NPP)}\\
A task is assigned the highest priority if it succeeds in locking a critical section.
The task is assigned its own priority when it releases the critical section.\\
Advantages:
\begin{itemize}
    \item Bounds(= limite) Priority Inversion and for a given task the bound is the maximal length
          of any single critical section belonging to lower priority tasks.
    \item It is deadlock free and limits the number of blocking of any task to one.
\end{itemize}
Disadvantages:
\begin{itemize}
    \item It allows low priority tasks to block high priority tasks.
\end{itemize}
\underline{Priority Inheritance Protocol (PIP)}\\
The idea is to elevate the priority of a low priority task to the highest priority of tasks
blocked by it. And resume its original priority when it exits the critical section.
This prevents medium-priority tasks from preempting lower priority tasks and thus
prolonging the blocking duration experienced by the higher-priority tasks.\\
Advantages:
\begin{itemize}
    \item Blocking time is bounded.
    \item Blocking time can be computed.
\end{itemize}
Disadvantages:
\begin{itemize}
    \item It is not deadlock free.
    \item Chain blocking can occur.
\end{itemize}
\underline{Highest locker priority protocol (HLP)}\\
Define the ceiling $C(S)$ of a critical section $S$ to be the highest priority
of all tasks that use $S$ during execution. Note that $C(S)$ can be calculated
statically (off-line). When it finishes with $S$, it sets its priority back to
what it was before.
\begin{table}[H]
    \centering
    \begin{tabularx}{\columnwidth}{
            @{}
            >{\hsize=0.95\hsize}X % 1st column, 5% smaller
            >{\hsize=0.95\hsize}X % 2nd column, 5% smaller
            >{\hsize=1.2\hsize}X  % 3rd column, 20% larger
            >{\hsize=0.95\hsize}X % 4th column, 5% smaller
            >{\hsize=0.95\hsize}X % 5th column, 5% smaller
            >{\hsize=0.95\hsize}X % 6th column, 5% smaller
            >{\hsize=0.95\hsize}X % 7th column, 5% smaller
            @{}
        } % 7 columns, third one is stretched more
        \toprule
        Algo- rithms & Chain blocking & Unnecessary blocking & Blocking instant & Deadlock prevention & Trans- parency & Imple- mentation \\
        \midrule
        NPP          & No             & Yes                  & arrival          & yes                 & Yes            & Easy             \\
        PIP          & Yes            & limited              & access           & no                  & Yes            & medium           \\
        HLP          & No             & Yes                  & arrival          & yes                 & No             & medium           \\
        \bottomrule
    \end{tabularx}
\end{table}
