\subsection*{C++}
Initializes a vector \texttt{v} with four integers and uses a range-based
for loop to print each element. The \texttt{auto} keyword automatically deduces
the element type (\texttt{int} in this case).
\begin{minted}{c++}
std::vector<int> v = {7, 5, 16, 8};
for (auto n : v) {
    std::cout << n << " ";
}
\end{minted}

Objects can be allocated on the stack or heap.\texttt{s1} and \texttt{s2} are stack-allocated
strings, while \texttt{s3} is a heap-allocated string. \texttt{s4} is a deep copy of \texttt{s2},
and \texttt{s5} is a shallow copy of \texttt{s3}. Changes to \texttt{s5} affect \texttt{s3}, as
shown by changing the first character to '@'. Heap memory is deallocated with \texttt{delete}.
\begin{minted}{c++}
std::string s1; //empty string
std::string s2("Hello"); //initialized string
std::string* s3 = new std::string("World"); //heap string
std::string s4 = s2; //deep copy
std::string* s5 = s3; //shallow copy
s5->at(0) = '@'; //s3 is also modified
delete s3; s3 = nullptr; //heap deallocation
delete s5; s5 = nullptr; 
\end{minted}

Declares two global variables. \texttt{globalVariable} can be accessed from any
part of the program, while \texttt{restrictedGlobalVariable} is \texttt{static},
limiting its scope to this file only.
\begin{minted}{c++}
int globalVariable = 9;
static int restrictedGlobalVariable = 10;
\end{minted}

Declares global constants and class constants. \texttt{kGlobalConstantExpr} is a
compile-time constant, while \texttt{kGlobalConstant} is a runtime constant. Inside
\texttt{ClassWithConstant}, \texttt{kConstantExpr} is a compile-time constant, and
\texttt{kConstant} is a runtime constant initialized with \texttt{kGlobalConstant * 2}.
\begin{minted}{c++}
static constexpr uint32_t kGlobalConstantExpr = 1;
static const uint32_t kGlobalConstant = rand();
class ClassWithConstant {
public:
    static constexpr uint32_t kConstantExpr = 2;
    static const uint32_t kConstant;
};
const uint32_t ClassWithConstant::kConstant = kGlobalConstant * 2;
\end{minted}

Demonstrates the use of fixed and dynamic arrays. \texttt{myArray1} is a
fixed-size array on the stack, while \texttt{myArray2} is a dynamically allocated array on
the heap. \texttt{myArray} is a dynamic array whose size is determined at runtime. The
\texttt{sizeof} operator is used to calculate the size of \texttt{myArray}.
\begin{minted}{c++}
static constexpr int arraySize = 30;
int myArray1[arraySize] = {0}; //fixed array on stack
int* myArray2 = new int[arraySize];  //dynamic array on heap
delete[] myArray2; //realease heap memory
myArray2 = nullptr; //reset pointer
int varSize = 5;
int myArray[varSize] = {0};
myArray[0] = 1;
myArray[1] = myArray[0];
int size = sizeof(myArray) / sizeof(int);
\end{minted}

Demonstrates the use of a \texttt{std::vector}, a dynamic array. It initializes
\texttt{v} with four integers, adds two more at the end, removes the last one, and accesses the
first and last elements. It also gets the size of the vector.
\begin{minted}{c++}
std::vector<int> v = {7, 5, 16, 8};
v.push_back(25); //add element to end
v.push_back(13); //add another element to end
v.pop_back(); //remove last element
std::cout << "First element is : " << v.front() << std::endl;
std::cout << "Last element is : " << v.back() << std::endl;
int size = v.size();
\end{minted}

Two ways to create a 2D array. The first part uses raw pointers to create
a jagged array, where each row has a different length. The second part uses \texttt{std::vector},
a dynamic array from the C++ Standard Library, to achieve the same result in a safer and simpler way.
\begin{minted}{c++}
char** array = new char*[base]; //jagged array
for (int i = 0; i < base; i++) {
    array[i] = new char[base - i + 1];
    for (int j = 0; j < base - i; j++){
        array[i][j] = '*';
    }
    array[i][base - i] = '\0';
}

std::vector<std::vector<char>> array(base); //2D vector
for (int i = 0; i < base; i++){
    array[i] = std::vector<char>(base - i, '*');
}
\end{minted}

Defines a \texttt{Point} class with two constructors, methods to move the point and
access its coordinates, and an operator overload for addition. The `+` operator is overloaded to
add the coordinates of two points.
\begin{minted}{c++}
class Point {
    public:
        Point() : _x(0), _y(0) { } // constructors
        Point(float x, float y) : _x(x), _y(y) { }
        ~Point() { // destructor
            std::cout << "Point destructor called" << std::endl;
        }
        void move(float dx, float dy) { // public methods
            _x += dx;
            _y += dy;
        }
        float x() const { return _x; }
        float y() const { return _y; }
        Point operator+(const Point& other) const {
            return Point(_x + other._x, _y + other._y);
        }
    private:
        float _x; // private data fields
        float _y;
};
\end{minted}

Demonstrates the concept of polymorphism. It defines a \texttt{Base} class and a
\texttt{Derived} class that inherits from \texttt{Base}. Both classes have a method \texttt{f()},
but in \texttt{Derived}, \texttt{f()} is overridden. When \texttt{f()} is called on a \texttt{Base}
reference or pointer that actually points to a \texttt{Derived} object, the \texttt{Derived}
version of \texttt{f()} is called due to the \texttt{virtual} keyword.
\begin{minted}{c++}
class Base {
public:
    virtual void f() {
        std::cout << "f() in Base class called" << std::endl;
    }
};
class Derived : public Base {
public:
    void f() override {
        std::cout << "f() in Derived class called" << std::endl;
    }
};
int main() {
    Base b; // Create a base instance
    Derived d; // Create a derived instance
    b.f(); // Prints base
    d.f(); // Prints derived
    Base& br = b; // The type of br is Base&
    Base& dr = d; // The type of dr is Base& as well
    br.f(); // Prints base
    dr.f(); // Prints derived (because Base::f() is declared as virtual)
    Base* bp = &b; // The type of bp is Base*
    Base* dp = &d; // The type of dp is Base* as well
    bp->f(); // Prints base
    dp->f(); // Prints derived (because Base::f() is declared as virtual)
    br.Base::f(); // prints base
    dr.Base::f(); // prints base
    return 0;
}
\end{minted}

Defines a \texttt{Buffer} class that allocates a block of memory when it is
constructed and releases that memory when it is destroyed. This is a common pattern for classes
that manage resources, which is often referred to as Resource Acquisition Is Initialization (RAII).
\begin{minted}{c++}
class Buffer {
public:
    Buffer(size_t size) : _size(size), _data(new char[size]) { }
    ~Buffer() { // destructor
        delete[] _data; // release the allocated memory
    }
private:
    size_t _size;
    char* _data;
};
\end{minted}

Demonstrates two ways of managing dynamic memory. The first part uses raw pointers to
allocate and deallocate memory manually, which can lead to memory leaks if not handled properly.
The second part uses \texttt{std::unique\_ptr}, a smart pointer that automatically deallocates the
memory when it's no longer needed, preventing memory leaks.
\begin{minted}{c++}
void function()
{
    char* pArray = new char[100]; // solution with raw pointer
    bool condition = false;
    if (condition) {
        // at this point, a memory leak arises since pArray is not released
        return;
    }
    delete [] pArray;
    pArray = nullptr;
    // solution with unique_ptr
    std::unique_ptr<char[]> array_ptr(new char[99]);
    if (condition) {
        // no memory leak since the destructor of array_ptr is called
        return;
    }
}
\end{minted}

Class templates allow for creating classes that can work with different data types and sizes.
The \texttt{Arithmetic<T>} class performs basic arithmetic operations on any type \texttt{T}. The
\texttt{Queue<T, queue\_sz>} class represents a queue of any type \texttt{T} with a size
\texttt{queue\_sz} known at compile time.
\begin{minted}{c++}
template <class T> class Arithmetic {
public:
    Arithmetic(T a, T b) : _a(a), _b(b) {}
    T add() { return _a + _b; }
    T sub() { return _a - _b; }
private:
    T _a, _b;
};
template <typename T, uint32_t queue_sz> class Queue {
public:
    uint32_t getSize() const { return queue_sz;}
private:
    T array[queue_sz];
};
int main() {
    Arithmetic<float> a2(1.5, 2.1);
    Queue<int, 10> intQueue;
    return 0;
}
\end{minted}